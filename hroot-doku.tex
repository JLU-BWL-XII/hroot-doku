\documentclass{article}
\usepackage[utf8]{inputenc}
\usepackage{hyperref}
\usepackage[T1]{fontenc}
\usepackage[english]{babel} 
\usepackage{graphicx}
\usepackage{xurl}
\usepackage{listings}
\usepackage{xcolor}

\usepackage{parskip}
\setlength{\parskip}{0pt} % 1ex plus 0.5ex minus 0.2ex}
\setlength{\parindent}{0pt}

% see https://www.overleaf.com/learn/latex/Code_listing
\definecolor{codegreen}{rgb}{0,0.6,0}
\definecolor{codegray}{rgb}{0.5,0.5,0.5}
\definecolor{codepurple}{rgb}{0.58,0,0.82}
\definecolor{backcolour}{rgb}{0.95,0.95,0.92}
\lstdefinestyle{mystyle}{
    backgroundcolor=\color{backcolour},   
    commentstyle=\color{codegreen},
    keywordstyle=\color{magenta},
    numberstyle=\tiny\color{codegray},
    stringstyle=\color{codepurple},
    basicstyle=\ttfamily\footnotesize,
    breakatwhitespace=false,         
    breaklines=true,                 
    captionpos=b,                    
    keepspaces=true,                 
    %numbers=left,                    
    numbersep=5pt,                  
    showspaces=false,                
    showstringspaces=false,
    showtabs=false,                  
    tabsize=2
}

\lstset{style=mystyle}

\title{hroot documentation}
\author{Tobias Weiss}
\date{\today}

\begin{document}

\maketitle

\section{Software install log}
Added software (name, date):
\begin{itemize}
    \item tmux, 2.2.22
    \item nginx, 2.2.22
    \item certbot, 2.2.22
    \item python3-certbot-nginx, 2.2.22
\end{itemize}

\section{docker}

All commands were executed for following containers: \verb|haproxy|, \verb|watchtower|, |hroot-app| and |hroot-db|. For the sake of brevity only one command is listed in the following.

\subsection{Disabled automatic restart for containers}

\begin{lstlisting}
sudo docker run -d --restart no haproxy
\end{lstlisting}

\subsection{Disabled docker service}

\begin{lstlisting}
sudo systemctl disable docker
\end{lstlisting}

\section{System update to bullseye}

Updated system to bullseye on 2.2.22.

New \verb|/etc/apt/sources.list|:
\begin{lstlisting}
deb http://ftp2.de.debian.org/debian bullseye main non-free contrib
deb-src http://ftp2.de.debian.org/debian bullseye main non-free contrib

deb http://security.debian.org/debian-security bullseye-security main contrib non-free
deb-src http://security.debian.org/debian-security bullseye-security main contrib non-free

deb http://deb.debian.org/debian bullseye-updates main contrib non-free
deb-src http://deb.debian.org/debian bullseye-updates main contrib non-free

\end{lstlisting}

\section{Unattended upgrades}

The package was already installed.
Run manually via \verb|sudo unattended-upgrade -d|

See doc: \url{https://wiki.debian.org/UnattendedUpgrades}

Changed \verb|/etc/apt/apt.conf.d/50unattended-upgrades|:
\begin{lstlisting}
Unattended-Upgrade::Mail "root";   
\end{lstlisting}

Added \verb|/etc/apt/apt.conf.d/20auto-upgrades|:
\begin{lstlisting}
APT::Periodic::Update-Package-Lists "1";
APT::Periodic::Unattended-Upgrade "1";
\end{lstlisting}

\section{arno firewall}
Reconfigure via \verb|dpkg-reconfigure -plow arno-iptables-firewall|

Added open ports (separated by whitespace) \verb|22 80 443|

\section{nginx}
Config \verb|/etc/nginx/sites-available/default|
\begin{lstlisting}
    server {

        # SSL configuration
        #
        # listen 443 ssl default_server;
        # listen [::]:443 ssl default_server;
        #
        # Note: You should disable gzip for SSL traffic.
        # See: https://bugs.debian.org/773332
        #
        # Read up on ssl_ciphers to ensure a secure configuration.
        # See: https://bugs.debian.org/765782
        #
        # Self signed certs generated by the ssl-cert package
        # Don't use them in a production server!
        #
        # include snippets/snakeoil.conf;

        root /var/www/html;

        # Add index.php to the list if you are using PHP
        index index.html index.htm index.nginx-debian.html;
    server_name experimente.wirtschaft.uni-giessen.de; # managed by Certbot


        location / {
                # First attempt to serve request as file, then
                # as directory, then fall back to displaying a 404.
                try_files $uri $uri/ =404;
        }

        # pass PHP scripts to FastCGI server
        #
        #location ~ \.php$ {
        #       include snippets/fastcgi-php.conf;
        #
        #       # With php-fpm (or other unix sockets):
        #       fastcgi_pass unix:/run/php/php7.4-fpm.sock;
        #       # With php-cgi (or other tcp sockets):
        #       fastcgi_pass 127.0.0.1:9000;
        #}

        # deny access to .htaccess files, if Apache's document root
        # concurs with nginx's one
        #
        #location ~ /\.ht {
        #       deny all;
        #}


    listen [::]:443 ssl ipv6only=on; # managed by Certbot
    listen 443 ssl; # managed by Certbot
    ssl_certificate /etc/letsencrypt/live/experimente.wirtschaft.uni-giessen.de/fullchain.pem; # managed by Certbot
    ssl_certificate_key /etc/letsencrypt/live/experimente.wirtschaft.uni-giessen.de/privkey.pem; # managed by Certbot
    include /etc/letsencrypt/options-ssl-nginx.conf; # managed by Certbot
    ssl_dhparam /etc/letsencrypt/ssl-dhparams.pem; # managed by Certbot

}
server {
    if ($host = experimente.wirtschaft.uni-giessen.de) {
        return 301 https://$host$request_uri;
    } # managed by Certbot


        listen 80 ;
        listen [::]:80 ;
    server_name experimente.wirtschaft.uni-giessen.de;
    return 404; # managed by Certbot
\end{lstlisting}

Restart via \verb|sudo nginx -t && sudo nginx -s reload|

\section{certbot}
Documentation: \\
\url{https://www.nginx.com/blog/using-free-ssltls-certificates-from-lets-encrypt-with-nginx/} 

Be aware the Debian package for nginx is called \verb|python3-certbot-nginx| (3 is missing in the guide)

Generate certificates with the NGINX plug-in:
\begin{verbatim}
sudo certbot --nginx -d experimente.wirtschaft.uni-giessen.de
\end{verbatim}

Added crontab entry via \verb|sudo crontab -e| for renewal:
\begin{lstlisting}
# m h  dom mon dow   command
0 5 * * * /usr/bin/certbot renew --quiet
\end{lstlisting}

\section{hroot installation}

\end{document}
