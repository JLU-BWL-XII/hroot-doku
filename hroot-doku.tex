\documentclass{article}
\usepackage[utf8]{inputenc}
\usepackage{hyperref}
\usepackage[T1]{fontenc}
\usepackage[english]{babel} 
\usepackage{graphicx}
\usepackage{xurl}
\usepackage{listings}
\usepackage{xcolor}

\usepackage{parskip}
\setlength{\parskip}{1pt} % 1ex plus 0.5ex minus 0.2ex}
\setlength{\parindent}{0pt}

% see https://www.overleaf.com/learn/latex/Code_listing
\definecolor{codegreen}{rgb}{0,0.6,0}
\definecolor{codegray}{rgb}{0.5,0.5,0.5}
\definecolor{codepurple}{rgb}{0.58,0,0.82}
\definecolor{backcolour}{rgb}{0.95,0.95,0.92}
\lstdefinestyle{mystyle}{
    backgroundcolor=\color{backcolour},   
    commentstyle=\color{codegreen},
    keywordstyle=\color{magenta},
    numberstyle=\tiny\color{codegray},
    stringstyle=\color{codepurple},
    basicstyle=\ttfamily\footnotesize,
    breakatwhitespace=false,         
    breaklines=true,                 
    captionpos=b,                    
    keepspaces=true,                 
    %numbers=left,                    
    numbersep=5pt,                  
    showspaces=false,                
    showstringspaces=false,
    showtabs=false,                  
    tabsize=2
}

\lstset{style=mystyle}

\title{hroot documentation}
\date{\today}

\begin{document}

\tableofcontents

\maketitle

\section{Contacts}
Main admin and responsible for the Server is Tobias Weiss.

Further server related contacts:
\begin{itemize}
    \item HRZ Hotline: 0641 99 13100 (Support time: working days 7:30-18:00)
    \item Herr Marc Lang, Marc.Lang@hrz.uni-giessen.de (HRZ VMware responsible, Backup): 0641 / 99-13145
    \item Herr Hartmut Weller, hartmut.weller@hrz.uni-giessen.de (HRZ): 0170 / 9107300
\end{itemize}

\section{Users}
\begin{itemize}
    \item penguin - HRZ admin user
    \item tux - chair admin user, password is documented in keepass
\end{itemize}

\section{Initial setup}
Inital setup took place on 30/04/2021.

The Ticket ID for the IP-Form was 27051.

\begin{lstlisting}
Hostname:         experimente
Domainname:       wirtschaft.uni-giessen.de

IP-Adresse:       134.176.7.57
Subnetzmaske:     255.255.255.0
Standardgateway:  134.176.7.1

Ethernetadresse:
DNS-Server:       134.176.2.5
DNS-Server:       134.176.2.7

Verwaltungsdaten:
Standort:         Heinrich-Buff-Ring 44
Raum:             SR 2
Hardware-Typ:     VIRTUAL-SERVER
Software-Typ:     LINUX-UBUNTU
\end{lstlisting}

\section{Access}
Access is granted to all chair members. Be aware of the chain of trust. Please consult the main in case of doubt.

You can use the IP \verb|134.176.7.57:22| to connect via SSH.

A \verb|~/.sshconfig| example:
\begin{lstlisting}
~/.sshconfig
Host experimente.wirtschaft.uni-giessen.de                                        
        User tux
\end{lstlisting}

\section{Software install log}
Added software packages (name, date):
\begin{itemize}
    \item certbot, 2.2.22
    \item libmariadb-deb, 2.2.22
    \item mariadb-client, 2.2.22
    \item mariadb-server, 2.2.22
    \item nginx, 2.2.22
    \item python3-certbot-nginx, 2.2.22
    \item tmux, 2.2.22
\end{itemize}

Manually installed (package, source, date):
\begin{itemize}
    \item rvm, \url{http://rvm.io/}, 2.2.22
\end{itemize}

\section{docker}

All commands were executed for following containers: \verb|haproxy|, \verb|watchtower|, |hroot-app| and |hroot-db|. For the sake of brevity only one command is listed in the following.

\subsection{Disabled automatic restart for containers}

\begin{lstlisting}
sudo docker run -d --restart no haproxy
\end{lstlisting}

\subsection{Disabled docker service}

\begin{lstlisting}
sudo systemctl disable docker
\end{lstlisting}

\subsection{Old hroot Dockerfile}
\begin{lstlisting}
FROM phusion/passenger-full:2.1.0

ENV HOME /root

RUN lsb_release -a

RUN rm -f /etc/service/nginx/down
ADD ./webapp.conf /etc/nginx/sites-enabled/webapp.conf
ADD ./hroot /home/app/hroot

RUN apt update && apt -y upgrade && apt install shared-mime-info

RUN bash -lc 'rvm install ruby-2.5.3'
RUN bash -lc 'rvm --default use ruby-2.5.3'
RUN bash -lc '/usr/bin/gem install bundler'
RUN bash -lc '(cd /home/app/hroot/; /usr/bin/bundle update mimemagic)'
RUN bash -lc '(cd /home/app/hroot/; /usr/bin/bundle install)'
RUN bash -lc '(cd /home/app/hroot/; /usr/bin/bundle exec rails assets:precompile)'

RUN bash -lc 'cp /usr/bin/ruby /usr/bin/ruby2.5.3'
RUN bash -lc 'chmod -R 0777 /home/app/hroot/'

RUN apt-get clean && rm -rf /var/lib/apt/lists/* /tmp/* /var/tmp/*

CMD ["/home/app/hroot/first_startup.sh"]
\end{lstlisting}

\section{System update to bullseye}

Updated system to bullseye on 2.2.22.

New \verb|/etc/apt/sources.list|:
\begin{lstlisting}
deb http://ftp2.de.debian.org/debian bullseye main non-free contrib
deb-src http://ftp2.de.debian.org/debian bullseye main non-free contrib

deb http://security.debian.org/debian-security bullseye-security main contrib non-free
deb-src http://security.debian.org/debian-security bullseye-security main contrib non-free

deb http://deb.debian.org/debian bullseye-updates main contrib non-free
deb-src http://deb.debian.org/debian bullseye-updates main contrib non-free

\end{lstlisting}

\section{Unattended upgrades}

The package was already installed.
Run manually via \verb|sudo unattended-upgrade -d|

See doc: \url{https://wiki.debian.org/UnattendedUpgrades}

Changed \verb|/etc/apt/apt.conf.d/50unattended-upgrades|:
\begin{lstlisting}
Unattended-Upgrade::Mail "root";   
\end{lstlisting}

Added \verb|/etc/apt/apt.conf.d/20auto-upgrades|:
\begin{lstlisting}
APT::Periodic::Update-Package-Lists "1";
APT::Periodic::Unattended-Upgrade "1";
\end{lstlisting}

\section{arno firewall}
Reconfigure via \verb|dpkg-reconfigure -plow arno-iptables-firewall|

Added open ports (separated by whitespace) \verb|22 80 443|

\section{nginx}
Config \verb|/etc/nginx/sites-available/default|
\begin{lstlisting}
    server {

        # SSL configuration
        #
        # listen 443 ssl default_server;
        # listen [::]:443 ssl default_server;
        #
        # Note: You should disable gzip for SSL traffic.
        # See: https://bugs.debian.org/773332
        #
        # Read up on ssl_ciphers to ensure a secure configuration.
        # See: https://bugs.debian.org/765782
        #
        # Self signed certs generated by the ssl-cert package
        # Don't use them in a production server!
        #
        # include snippets/snakeoil.conf;

        #root /var/www/html;
        root /home/tux/hroot/public;

        passenger_enabled on;
        passenger_app_env production;

        # Add index.php to the list if you are using PHP
        index index.html index.htm index.nginx-debian.html;
    server_name experimente.wirtschaft.uni-giessen.de; # managed by Certbot

        location / {
                # First attempt to serve request as file, then
                # as directory, then fall back to displaying a 404.
                try_files $uri $uri/ =404;
        }

        # pass PHP scripts to FastCGI server
        #
        #location ~ \.php$ {
        #       include snippets/fastcgi-php.conf;
        #
        #       # With php-fpm (or other unix sockets):
        #       fastcgi_pass unix:/run/php/php7.4-fpm.sock;
        #       # With php-cgi (or other tcp sockets):
        #       fastcgi_pass 127.0.0.1:9000;
        #}

        # deny access to .htaccess files, if Apache's document root
        # concurs with nginx's one
        #
        #location ~ /\.ht {
        #       deny all;
        #}


    listen [::]:443 ssl ipv6only=on; # managed by Certbot
    listen 443 ssl; # managed by Certbot
    ssl_certificate /etc/letsencrypt/live/experimente.wirtschaft.uni-giessen.de/fullchain.pem; # managed by Certbot
    ssl_certificate_key /etc/letsencrypt/live/experimente.wirtschaft.uni-giessen.de/privkey.pem; # managed by Certbot
    include /etc/letsencrypt/options-ssl-nginx.conf; # managed by Certbot
    ssl_dhparam /etc/letsencrypt/ssl-dhparams.pem; # managed by Certbot

}
server {
    if ($host = experimente.wirtschaft.uni-giessen.de) {
        return 301 https://$host$request_uri;
    } # managed by Certbot


        listen 80 ;
        listen [::]:80 ;
    server_name experimente.wirtschaft.uni-giessen.de;
    return 404; # managed by Certbot
\end{lstlisting}

Restart via \verb|sudo nginx -t && sudo nginx -s reload|

\section{certbot}
Documentation: \\
\url{https://www.nginx.com/blog/using-free-ssltls-certificates-from-lets-encrypt-with-nginx/}

Be aware the Debian package for nginx is called \verb|python3-certbot-nginx| (3 is missing in the guide)

Generate certificates with the NGINX plug-in:
\begin{verbatim}
sudo certbot --nginx -d experimente.wirtschaft.uni-giessen.de
\end{verbatim}

Added crontab entry via \verb|sudo crontab -e| for renewal:
\begin{lstlisting}
# m h  dom mon dow   command
0 5 * * * /usr/bin/certbot renew --quiet
\end{lstlisting}

\section{Postfix}
Change \verb|/etc/postfix/canonical| for mail redirection.
Do not forget to update the map after editing via \verb|sudo postmap /etc/postfix/canonical|.
\begin{lstlisting}
#root    linux-admins@hrz.uni-giessen.de
root    tobias.weiss@wirtschaft.uni-giessen.de
tux    tobias.weiss@wirtschaft.uni-giessen.de
\end{lstlisting}


\section{hroot}
Source: \url{https://github.com/wiso-forschungslabor/hroot}

Documentation: \url{https://github.com/wiso-forschungslabor/hroot-documentation/wiki/}

\subsection{rails install}
\begin{lstlisting}
sudo gpg --recv-keys 409B6B1796C275462A1703113804BB82D39DC0E3 7D2BAF1CF37B13E2069D6956105BD0E739499BDB
curl -sSL https://get.rvm.io | sudo bash -s stable
sudo usermod -a -G rvm tux
source /etc/profile.d/rvm.sh
rvm info
rvm install 2.5.3
rvm gemset create hroot
rvm use 2.5.3@hroot
ruby --version
\end{lstlisting}

\subsection{Mariadb install}
\begin{lstlisting}
sudo apt-get install mysql-server mysql-client libmariadb-dev
sudo mysql_secure_installation
\end{lstlisting}

Note: Root PW was changed to something else than system login!

After secure installation log into mysql via \verb|mysql -u root -p|.
Create databases and hroot user:
\begin{lstlisting}
CREATE DATABASE hroot_development;
CREATE DATABASE hroot_test;
CREATE DATABASE hroot_production;
CREATE USER 'hroot'@'localhost' IDENTIFIED BY 'password';
GRANT ALL ON hroot_development.* TO 'hroot'@'localhost'; 
GRANT ALL ON hroot_test.* TO 'hroot'@'localhost'; 
GRANT ALL ON hroot_production.* TO 'hroot'@'localhost';
FLUSH PRIVILEGES;
\end{lstlisting}

Deleted plain password in \verb|~/.mysql_history| afterwards.

For further docs see \url{https://dev.mysql.com/doc/refman/8.0/en/grant.html}

\subsection{hroot install}
Forked hroot to GitHub account.
To clone the git repo an access token is needed.
Create one GitHub settings $\rightarrow$ Developer Settings $\rightarrow$ Personal access tokens
\begin{lstlisting}
cd /home/tux
git clone https://github.com/JLU-BWL-XII/hroot.git
\end{lstlisting}

\begin{lstlisting}
gem install bundler
\end{lstlisting}

Fix \verb|mimemagic| issue by editing the Gem file, add:
\begin{lstlisting}
    gem 'mimemagic', github: 'mimemagicrb/mimemagic', ref: '01f92d86d15d85cfd0f20dabd025dcbd36a8a60f'
\end{lstlisting}
Now change bundle to use HTTPS:
\begin{lstlisting}
bundle config hithub.https true
\end{lstlisting}

See \url{https://stackoverflow.com/questions/66919504/your-bundle-is-locked-to-mimemagic-0-3-5-but-that-version-could-not-be-found}

Install gems:
\begin{lstlisting}
bundle install
\end{lstlisting}

Create databse config \verb|config/database.yml|:
\begin{lstlisting}
    development:
    adapter: mysql2
    encoding: utf8
    reconnect: false
    database: hroot_development
    pool: 5
    username: hroot
    password: insert_your_database_password_here
    host: localhost
  
  # Warning: The database defined as "test" will be erased and
  # re-generated from your development database when you run "rake".
  # Do not set this db to the same as development or production.
  test:
    adapter: mysql2
    encoding: utf8
    reconnect: false
    database: hroot_test
    pool: 5
    username: hroot
    password: insert_your_database_password_here
    host: localhost
  
  production:
    adapter: mysql2
    encoding: utf8
    reconnect: false
    database: hroot_production
    pool: 5
    username: hroot
    password: insert_your_database_password_here
    host: localhost
\end{lstlisting}

Create \verb|secret_key_base| via \verb|EDITOR=vim rails credentials:edit|

Create tables and admin user:
\begin{lstlisting}
rails db:migrate
rails hroot:create_admin
RAILS_ENV=production rails db:migrate
RAILS_ENV=production rails hroot:create_admin
\end{lstlisting}

Create \verb|config/secrets.yml| and generate two keys via \verb|rails secret|.

Run local server and tests:
\begin{lstlisting}
rails server
rails db:test:prepare
rails test
\end{lstlisting}

Application configs are in \verb|config/environments| (one per environment).

\subsection{Passenger}
Documentation: \url{https://www.phusionpassenger.com/docs/tutorials/deploy_to_production/installations/oss/ownserver/ruby/nginx/}

\begin{lstlisting}
sudo apt-key adv --keyserver hkp://keyserver.ubuntu.com:80 --recv-keys 561F9B9CAC40B2F7
sudo sh -c 'echo deb https://oss-binaries.phusionpassenger.com/apt/passenger bullseye main > /etc/apt/sources.list.d/passenger.list'
sudo apt-get update
if [ ! -f /etc/nginx/modules-enabled/50-mod-http-passenger.conf ]; then sudo ln -s /usr/share/nginx/modules-available/mod-http-passenger.load /etc/nginx/modules-enabled/50-mod-http-passenger.conf ; fi
sudo ls /etc/nginx/conf.d/mod-http-passenger.conf
sudo service nginx restart
sudo /usr/bin/passenger-config validate-install
sudo /usr/sbin/passenger-memory-stats
\end{lstlisting}

Precompile production assets:
\begin{lstlisting}
RAILS_ENV=production rails assets:precompile
\end{lstlisting}

\subsection{hroot cronjobs}
Generate periodic tasks via \verb|whenever| and add them to \verb|crontab -e|:
\begin{lstlisting}
0,5,10,15,20,25,30,35,40,45,50,55 * * * * /bin/bash -l -c 'cd /home/tux/hroot && bundle exec bin/rails runner -e production '\''Task.run_tasks'\''' > /dev/null 2>&1

0 0 * * * /bin/bash -l -c 'cd /home/tux/hroot && bundle exec bin/rails runner -e production '\''LoginCode.cleanup'\''' > /dev/null 2>&1

0 0 * * * /bin/bash -l -c 'cd /home/tux/hroot && bundle exec bin/rails runner -e production '\''Task.send_warnings'\''' > /dev/null 2>&1

0 0 * * * /bin/bash -l -c 'cd /home/tux/hroot && bundle exec bin/rails runner -e production '\''OneYearReminders.send_one_year_warnings1'\''' > /dev/null 2>&1

0 0 * * * /bin/bash -l -c 'cd /home/tux/hroot && bundle exec bin/rails runner -e production '\''OneYearReminders.send_one_year_warnings2'\''' > /dev/null 2>&1

0 0 * * * /bin/bash -l -c 'cd /home/tux/hroot && bundle exec bin/rails runner -e production '\''OneYearReminders.disable_after_one_year'\''' > /dev/null 2>&1
\end{lstlisting}

\subsection{Logo}
Replaced JLU logo at \verb|app/assets/images/mainlogo.png|

\section{Monitoring}
\subsection{Web server monitoring}
Web server is externally monitored via a private host (tobias-weiss.org) by means a simple ping script.
Ask Tobias Weiss for details.

\subsection{Disk and memory monitoring}
The simple \verb|home/tux/monitoring/disk-monitor.py| and \verb|memory-monitor.py| scripts are used.
They alarm if disk is filled by more than 80\% or RAM less free than 200 MB.
Note that the config resides in \verb|conf.py|.
They are scheduled via \verb|/etc/crontab|:
\begin{lstlisting}
42 * * * * root free -m | python2.7 /home/tux/monitoring/memory-monitor.py
42 * * * * root df -h | python2.7 /home/tux/monitoring/disk-monitor.py
\end{lstlisting}

\section{Backup}
\subsection{System Backup}
Storage policy:daily full (at 20:16), kept 60 days on disk and one year on tape.
Herr Marc Lang creates a daily backup.

\subsection{MySQL Backup}
Stored in \verb|/home/tux/backup|

Changed access rights to folder via \verb|chmod o-rx \verb|/home/tux/backup/|

Storage policy: daily full, 30 days

Daily database backup is scheduled via \verb|sudo crontab -e|:
\begin{lstlisting}   
24 3 * * * /usr/local/sbin/hroot-db-backup.sh 
\end{lstlisting}

The script is located at \verb|/usr/sbin/hroot-db-backup.sh|. Only root can read and execute.
\begin{lstlisting}
#!/bin/bash
# backup script for the hroot production database
# scheduled via
# sudo crontab -e

DB=hroot_production
PATH=/home/tux/backup
USER=hroot
PASS=hrootDatabasePW
DATE=`/usr/bin/date +%Y-%m-%d`

echo "$DB database backup $DATE" 
/usr/bin/mysqldump -u $USER -p$PASS $DB > $PATH/$DB-$DATE.sql
echo "exit status: $?" 

echo "Change access rights"
/usr/bin/chmod o-rwx $PATH/*
echo "exit status: $?"

echo "deleting backups older than 30 days"
/usr/bin/find /home/tux/backup -name "*.sql" -type f -mtime +30 -delete
echo "exit status: $?" 

echo "folder content:"
/usr/bin/ls -lah $PATH

echo "folder size:"
/usr/bin/du -sh $PATH
\end{lstlisting}

\end{document}
